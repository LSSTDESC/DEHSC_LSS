\documentclass[a4paper,11pt]{article}
\pdfoutput=1 % if your are submitting a pdflatex (i.e. if you have
             % images in pdf, png or jpg format)

\usepackage{jcappub}
\usepackage{graphicx}
\usepackage{dcolumn}
\usepackage{amssymb,amsmath,bm}
\usepackage{color}
\usepackage[dvipsnames]{xcolor}
%\usepackage[colorlinks,linkcolor=red,citecolor=blue,urlcolor=blue ]{hyperref}
\usepackage[utf8]{inputenc}
\usepackage{lipsum}
\newcommand{\nv}{\vec{\theta}}
\newcommand{\todo}[1]{{\bf TODO: #1}}

\usepackage[T1]{fontenc} % if needed
\usepackage{natbib}
\bibliographystyle{JHEP}
\title{Tomographic galaxy clustering with the Subaru Hyper Suprime-Cam first year public data release}

\author[a,1]{Alonso D.}
\author[b]{Awan H.}
\author[b]{Broussard A.}
\author[b]{Gawiser E.}
\author[b]{Gomes Z.}
\author[b]{Lupton R.}
\author[b]{Mandelbaum R.}
\author[b]{Miyatake H.}
\author[b]{Nicola A.}
\author[b]{S\'anchez F.J.}
\author[b]{Slosar A.}

\affiliation[a]{Department of Physics, University of Oxford, Keble Road, Oxford, OX1 3RH, UK}
\affiliation[b]{Some other sunnier place}
\emailAdd{david.alonso@physics.ox.ac.uk}

\abstract{\lipsum[1]}

\begin{document}
\maketitle
\flushbottom

\section{Introduction}\label{sec:intro}
  \lipsum[1]

\section{Theory}\label{sec:theory}
  \subsection{Projected quantities and power spectra}\label{ssec:theory.cellpk}
    \todo{cites everywhere}
    Our main observable is the projected overdensity of galaxies $\delta^i_g(\nv)$ as a function of sky position $\nv$ in a given redshift bin labelled by $i$. This is related to the 3D galaxy overdensity $\Delta_g$ through
    \begin{equation}
      \delta^i_g(\nv)=\int dz\,p^i(z)\,\Delta_g\left(t(z),\chi(z)\nv\right).
    \end{equation}
    Here $t(z)$ and $\chi(z)$ are the cosmic time and radial comoving distance as a function of redshift, and $p_i(z)$ is the redshift bin window function, given by the true redshift distribution of objects in the bin normalized to unit area.
  
    Given the small size of the sky patches covered by the HSC DR1, we will adopt the flat sky approximation for simplicity, in which case $\nv$ is a 2D vector. It is common to decompose $\delta^i_g(\nv)$ into its Fourier coefficients
    \begin{equation}
      a^i_{\bf l}\equiv \int\frac{d\theta^2}{2\pi}e^{-i{\bf l}\cdot\nv}\delta^i_g(\nv).
    \end{equation}
    The variance of the Fourier coefficients is the so-called angular power spectrum $\langle a^i_{\bf l}a^{j*}_{{\bf l}'}\rangle = C^{ij}_\ell\,\delta^{\cal D}({\bf l}-{\bf l}')$, where $\delta^{\cal D}$ is the 2D Dirac delta function. The 3D power spectrum $P_{gg}(z,{\bf k})$ is defined analogously for the 3D Fourier coefficients of $\Delta_g$. Both quantities are related to each other through:
    \begin{equation}
      C^{ij}_\ell = \int dz\,\frac{H(z)}{\chi^2(z)} p^i(z)p^j(z)\,P_{gg}\left(z,k=\frac{\ell+1/2}{\chi(z)}\right),
    \end{equation}
    where $H(z)$ is the expansion rate at redshift $z$, and we have used the so-called Limber approximation.
  
  \subsection{Magnification bias}\label{ssec:theory.mag}
    \todo{cites everywhere}
    Gravitational lensing alters the observed galaxy fluxes and distorts their observed positions. The combined effect, is a position-dependent modulation of the number density that distorts the clustering pattern.
    In the presence of this effect, the equations in the previous section get slightly modified. The projected galaxy overdensity becomes:
    \begin{equation}
      \delta^i_g(\nv)=\int dz\,\left[p^i(z)\,\Delta_g+\frac{W_\mu^i(z)}{H(z)}\nabla_\theta^2\nabla^{-2}\Delta_m\right],
    \end{equation}
    where $\Delta_m$ is the 3D matter overdensity (we have omitted the dependence on $t$ and $\chi\nv$ for brevity) and $W_\mu$ is the magnification kernel
    \begin{equation}
      W_\mu(z)=\frac{3H_0^2\Omega_M(1+z)}{2}\int_z^\infty dz'\,p^i(z)\,\left(5s(z')-2\right)\,\frac{\chi(z')-\chi(z)}{\chi(z)\chi(z')}.
    \end{equation}
    Here, $H_0\equiv H(z=0)$, $\Omega_M$ is the cosmological matter fraction, and $s$ is the logarithmic slope of the cumulative apparent magnitude distribution:
    \begin{equation}
      s\equiv\frac{\partial \log_{10}N(<m)}{\partial m}.
    \end{equation}
    The angular power spectrum is then given by:
    \begin{equation}
      C^{ij}_\ell = \int dz\,\frac{H(z)}{\chi^2(z)}\left[p^ip^j\,P_{gg}+\left(p^iW_\mu^j+p^jW_\mu^i\right)\frac{\ell(\ell+1)}{Hk^2}P_{gm}+W_\mu^iW_\mu^j\left(\frac{\ell(\ell+1)}{Hk^2}\right)^2P_{mm}\right],
    \end{equation}
    where $P_{mm}$ is the 3D power spectrum of matter fluctuations, $P_{gm}$ is the galaxy-matter cross-spectrum and we have omitted the dependence of all quantities on $z$ or $k=(\ell+1/2)/\chi$ for brevity.

  \subsection{Halo Occupation Distribution}\label{ssec:theory.hod}
    In order to model $P_{gg}(z,k)$ we use a halo occupation distribution (HOD) model \todo{cites everywhere}. In this halo model-based presciption we model the galaxy content of dark matter haloes as a function of halo mass. Details about HOD parametrizations can be found in \todo{cite}. In short, the galaxy power spectrum receives contributions from the so-called 1-halo and 2-halo terms:
    \begin{equation}
      P_{gg}(z,k) = P_{gg,{\rm 1h}}(z,k) + P_{gg,{\rm 2h}}(z,k),
    \end{equation}
    where
    \begin{align}
      & P_{gg,{\rm 1h}}(k)=\frac{1}{\bar{n}_g^2} \int dM\,\frac{dn}{dM} \bar{N}_c\,\left[\bar{N}_s^2u_s^2(k)+2\bar{N}_su_s^2(k)\right],\\
      & P_{gg,{\rm 1h}}(k)=\left(\frac{1}{\bar{n}_g} \int dM\,\frac{dn}{dM} \bar{N}_c\,\left[1+\bar{N}_su_s(k)\right]\right)^2\,P_{\rm lin}(k).\\
    \end{align}
    Here, $M$ represents halo mass, $dn/dM$ is the halo mass function, $\bar{N}_c(M)$ and $\bar{N}_s(M)$ are the mean number of central and satellite galaxies respectively, $u_s(k)$ is the Fourier transform of the normalized density profile of satellite galaxies, $P_{\rm lin}(k)$ is the linear matter power spectrum and $\bar{n}_g$ is the total mean galaxy density, given by
    \begin{equation}
      \bar{n}_g=\int dM\,\frac{dn}{dM}\bar{N}_c(M)\left[1+\bar{N}_s(M)\right].
    \end{equation}

    Following \todo{cite}, we parametrise the number of centrals and satellites as a function of mass as:
    \begin{align}
      &\bar{N}_c(M)=\frac{1}{2}\left[1+{\rm erf}\left(\frac{\log(M/M_{\rm min})}{\sigma_{{\rm ln}M}}\right)\right],\\
      &\bar{N}_s(M)=\Theta(M-M_0)\left(\frac{M-M_0}{M_1'}\right)^\alpha,
    \end{align}
    where $\Theta(x)$ is the Heavyside step function. Furthermore, we assume that $N_c$ follows a Bernoully distribution with probability $p=\bar{N}_c$, and that the number of satellites is Poisson-distributed with mean $\bar{N}_s$. Finally, we model the distribution of satellites to follow that of the dark matter, and therefore $u_s$ is a Navarro-Frenk-White profile, given by:
    \begin{equation}
      u_s(k|M)=\frac{\sin x\left[{\rm Si}\left((1+c)\,x\right)-{\rm Si}(x)\right]+\cos x\left[{\rm Ci}\left((1+c)x\right)-{\rm Ci}(x)\right]-\frac{\sin(cx)}{(1+c)x}}{{\bf ln}(1+c)-\frac{c}{1+c}},
    \end{equation}
    where $x=k R_\Delta/c$, $R_\Delta$ is the halo radius, $c=c(M)$ is the concentration parameter, and ${\rm Si}/{\rm Ci}$ are the sine and cosine integral functions. We define $R_\Delta$ as the radius that encloses $\Delta=200$ times the background matter density. The concentration-mass relation $c(M)$ also depends on the choice of $\Delta$, and we follow the parametrization of Duffy et al. \todo{cite}.

    \todo{Possibly extend this to describe $P_{gm}$}
    
  \subsection{Covariance matrices}\label{ssec:theory.ssc}
    \todo{Possibly also describe theory modelling behind covariance. }
    \lipsum[3]

\section{Data}\label{sec:data}
  \todo{Describe DR1 and sample selection \citep{2018arXiv180909148H}. Describe redshift binning.}
  \lipsum[3]

\section{Methods}\label{sec:methods}
  \subsection{Systematics maps}\label{ssec:methods.syst}
    \todo{Describe how we generate systematic maps}
    \lipsum[3]

  \subsection{Redshift distributions}\label{ssec:methods.nz}
    \todo{Describe methods used to estimate $N(z)$.}
    \lipsum[5]

  \subsection{Angular power spectra}\label{ssec:methods.cell}
    \todo{Describe how the $C_\ell$ estimator works, possibly also deprojection}
    \lipsum[6]

  \subsection{Covariance matrices}\label{ssec:methods.covar}
    \todo{Describe covariance. Describe coaddition.}
    \lipsum[7]
    
\section{Results}\label{sec:results}
  \lipsum[8]
  \begin{figure}
    \centering
    \includegraphics[width=0.9\columnwidth]{figures/token_plot}
    \caption{Blah}\label{fig:token}
  \end{figure}
  \todo{Things to discuss:}
  \begin{enumerate}
    \item Impact of systematics. Discuss star contamination in more detail.
    \item Checks we've done: consistency between fields ($C_\ell$ and $N_{\rm gal}$), different masks, different levels of contaminant deprojection, different covariances.
    \item Effects of deprojection.
    \item Final contours.
    \item ...
  \end{enumerate}

\section{Discussion}\label{sec:discussion}
  \lipsum[9]

\acknowledgments

We thank the Backstreet Boys for useful comments and discussions.

\bibliography{bibliography}

\end{document}
